\documentclass[aspectratio=169]{beamer}
\usepackage{ifxetex}

\ifxetex%
	\usepackage{polyglossia}
%	\setdefaultlanguage{russian}
	\setdefaultlanguage{english}
\else
	\usepackage[utf8]{inputenc}
	\usepackage[T1]{fontenc}
	\usepackage[english]{babel}
%	\usepackage[english,russian]{babel}
\fi

\usetheme[]{SaintPetersburg}

\title{Saint Petersburg \LaTeX~Beamer theme}
\author{Ivan Gankevich}
\institute{Saint Petersburg State University}
\date{2017}

\begin{document}

\frame{\titlepage}

\begin{frame}{Outline}
	\tableofcontents
\end{frame}

\section{Environments}

\frame{\sectionpage}

\begin{frame}
	\frametitle{Theorems}
    \begin{columns}[T]
        \begin{column}{0.45\textwidth}
			\begin{example}\end{example}
			\begin{examples}\end{examples}
			\begin{definition}\end{definition}
			\begin{definitions}\end{definitions}
        \end{column}
        \begin{column}{0.45\textwidth}
			\begin{theorem}[Fermat]\end{theorem}
			\begin{proof}[Proof\nopunct]\end{proof}
			\begin{corollary}\end{corollary}
			\begin{lemma}\end{lemma}
			\begin{fact}\end{fact}
        \end{column}
	\end{columns}
\end{frame}

\begin{frame}
    \frametitle{Blocks}
    \begin{columns}[T]
        \begin{column}{0.45\textwidth}
            \begin{block}{Block}\end{block}
            \begin{exampleblock}{Example block}\end{exampleblock}
            \begin{alertblock}{Alert block}\end{alertblock}
        \end{column}
        \begin{column}{0.45\textwidth}
        \end{column}
    \end{columns}
\end{frame}

\section{Fonts}

\begin{frame}[fragile]
	\frametitle{Font shapes}
	\begin{center}
		\begin{tabular}{lll}
			\textup{\textsf{Saint Petersburg}} &
			\textup{\textrm{Saint Petersburg}} &
			\textup{\texttt{Saint Petersburg}} \\
			\textit{\textsf{Saint Petersburg}} &
			\textit{\textrm{Saint Petersburg}} &
			\textit{\texttt{Saint Petersburg}} \\
			\textbf{\textsf{Saint Petersburg}} &
			\textbf{\textrm{Saint Petersburg}} &
			\textbf{\texttt{Saint Petersburg}} \\
			\textit{\textbf{\textsf{Saint Petersburg}}} &
			\textit{\textbf{\textrm{Saint Petersburg}}} &
			\textit{\textbf{\texttt{Saint Petersburg}}} \\
		\end{tabular}
	\end{center}
\end{frame}

\begin{frame}
	\frametitle{Equations}
	\begin{equation*}
		\zeta_{\vec{\nu}} = \sum\limits_{\vec{u}=\vec{0}}^{\vec{N}}
			\Phi_{\vec{u}}
			\zeta_{\vec{\nu} - \vec{u}} + \epsilon_{\vec{\nu}}
	\end{equation*}
\end{frame}

\begin{frame}
	\frametitle{References}
	\nocite*{}
	\bibliographystyle{plain}
	\bibliography{refs}
\end{frame}

\end{document}
